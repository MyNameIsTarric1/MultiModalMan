\section{Requirements}

\subsection{Overview}
The following tables specify the essential capabilities (\textit{Functional}) and quality targets (\textit{Non-Functional}) that \textit{MultiModalMan} must meet to deliver an inclusive, responsive and maintainable gaming experience.

% --------------------------------------------------
\subsection{Functional Requirements}

\begin{table}[h]
\centering
\caption{Functional requirements}
\begin{tabularx}{\linewidth}{@{}lX@{}}
\toprule
\textbf{ID} & \textbf{Description} \\ \midrule
F1 & The user can start a new game at any time. \\
F2 & The system selects a random word from an internal dictionary. \\
F3 & Letters can be guessed through voice commands, hand gestures or keyboard input. \\
F4 & The interface shows the current word state and the number of wrong attempts. \\
F5 & The system announces the outcome (win or loss) when the game ends. \\
F6 & A new round can be launched without restarting the application. \\ \bottomrule
\end{tabularx}
\end{table}

% --------------------------------------------------
\subsection{Non-Functional Requirements}

\begin{table}[h]
\centering
\caption{Non-functional requirements}
\begin{tabularx}{\linewidth}{@{}lX@{}}
\toprule
\textbf{ID} & \textbf{Description} \\ \midrule
NF1 & The user interface must be clear and readable for all age groups. \\
NF2 & The game runs on Windows, macOS and Linux with a standard webcam and microphone. \\
NF3 & Average recognition latency for voice or gesture input is \(\leq\) 2 s. \\
NF4 & Gesture recognition covers the full 26-letter English alphabet\footnote{Or 21 letters for Italian; adjust to the target language set.}. \\
NF5 & The code base is modular and documented to support future extensions. \\
NF6 & The game logic remains consistent when inputs are invalid or ambiguous. \\ \bottomrule
\end{tabularx}
\end{table}
